\documentclass[letterpaper,12pt]{article}

\usepackage[spanish]{babel}
\usepackage[utf8]{inputenc}

\usepackage{graphicx} % graficos
\usepackage{amsmath}
\begin{document}
	
	\begin{titlepage}
		
		\begin{center}
			\vspace*{-1in}
			\begin{figure}[htb]
				\begin{center}
					\includegraphics[width=8cm]{Imagen.jpeg}
				\end{center}
			\end{figure}
		
			\vspace*{0.15in}
			\begin{Large}
				\begin{center}
					\textbf{SISTEMA DE AYUDA AL DIAGNÓSTICO DE
TRASTORNOS DEL DESARROLLO EN
ENTORNOS ESCOLARES}
				\end{center}
			\end{Large}
			\vspace*{0.3in}
			\begin{large}
				ESCUELA SUPERIOR DE INFORMÁTICA\\
			\end{large}
			\vspace*{0.3in}
			\rule{150mm}{0.1mm}\\
			\vspace*{0.6in}
			\begin{large}
				\begin{flushleft}
					{\normalsize Asignatura: Sistemas Basados en el Conocimiento}
				\end{flushleft}
				\begin{flushleft}
					{\normalsize Autores: María Blanco González-Mohino\\} 
				\end{flushleft}
				\begin{flushleft}
					{\normalsize Fecha: 18 de Febrero de 2021}
				\end{flushleft}
			\end{large}
		\end{center}
		\end{titlepage}
		\tableofcontents
		\newpage
		\section{Descripción}
		zcxccsda
		El objetivo principaldel proyecto es desarrollar un mecanismo computacional capaz de identificar si un alumno/a sufre algún tipo de trastorno del desarrollo, este sistema estará enfocado para su uso por profesores ya que son estos quienes en la mayoría de casos advierten sobre una posible alteración.
El sistema contendrá una serie de reglas que desembocarán en un diagnostico hacia un alumno; también se incluirá una serie de test utilizados para medir el desarrollo que podrían ser de utilidad.

	\subsection{Alcance y límites}
	Este sistema distinguirá entre varios tipos de trastornos: \textit{trastornos por déficit de atención, trastornos generalizado del aprendizaje, trastorno del espectro autista y deficiencia cognitiva.}
	
	No están incluidos cualquier otro tipo de trastorno del desarrollo como podrían ser la deficiencias sensoriales, ni estan incluidos trastornos pertenecientes a categorias diferentes como serían los trastornos conductuales.
	
	\section{Estudio de viabilidad}
	En este apartado se va a analizar la viabilidad del proyecto para justificar su realizacion. Se aplicará el test de Slager, en el que se calificarán una serie de características divididas en 4 dimensiones \textbf{(Plausibilidad, Justificación, Adecuación,
Éxito).} Cada tarea/característica especificada puede ser esencial o deseable, a cada una de las características se le asignará un valor y un peso, el valor de las características esenciales no puede ser menor a 7, de otro modo el sistema no sería viable. Según su importancia relativa, el peso de cada característica está entre 0 y 10.
	\subsection{Evaluacion de la aplicación candidata}
	Cada una de estas dimensiones se definirá, se le asignarán pesos a cada característica y se evaluará cada dimensión en cada respectivo subapartado.
A continuación se presenta una tabla a modo de leyenda para el entendimiento de cada tabla perteneciente a la aplicación candidata.

\begin{tabular}{|c|c|c|}
\hline 
Identificador & Significado & Rango \\ 
\hline
CAT & Categoría & ----- \\ 
\hline 
EX & Expertos & • \\ 
\hline 
TA & Tarea & • \\ 
\hline 
DU & Directivos o usuarios & • \\ 
\hline 
E & Esencial & • \\ 
\hline 
D & Deseable & • \\ 
\hline 
Pi & Característica de la dimensión de \textbf{Plausibilidad} & P1 ... P10 \\ 
\hline 
jI & Característica de la dimensión de Justificación & J1 ... J7 \\ 
\hline 
Ai & Característica de la dimensión de Adecuación & A1 ... A12 \\ 
\hline 
Ei & Característica de la dimensión de Éxito & E1 ... E17 \\ 
\hline 
\end{tabular} \\

Tabla para el entendimiento de la evaluación: \\

\begin{center}
	\begin{tabular}{|c|c|}
\hline 
Identificador & Significado \\ 
\hline 
CV & Valor global de una aplicación según la dimensión \\ 
\hline 
Vu & Valor umbral del sistema \\ 
\hline 
Vp & Valor de una característica según la dimensión \\ 
\hline 
Pp & Peso de la característica según la dimensión \\ 
\hline 
Vp & Valor de una característica según la dimensión \\ 
\hline 
\end{tabular} \\
\end{center}
\newpage
\subsubsection{Plausabilidad}
Primera dimensión, un sistema es plausible si la tarea no requiere de sentido común y contamos con suficientes expertos con un índice de cooperatividad adecuado de modo que la tarea no nos resulte demasiado compleja. \\

Obtención de valores para el cálculo de la plausibilidad: \\
\begin{tabular}{|c|c|c|c|p{7.3 cm}|c|}
\hline 
CAT & IDEN. CAT & PESO & VALOR & DENOMINACIÓN DE LA CARACTERÍSITCA & TIPO \\ 
\hline 
EX & P1 & 10 & 8 & Existen expertos. Se tomá como expertos al profesorado del CEIP Albuera, Daimiel(Ciudad Real) & E \\ 
\hline 
EX & P2 & 10 & 9 & El experto asignado es genuino. & E \\ 
\hline 
EX & P3 & 8 & 10 & El experto es cooperativo. & D \\ 
\hline 
EX & P4 & 7 & 6 & El experto es capaz de articular sus métodos pero no categoriza. & D \\ 
\hline 
TA & P5 & 10 & 10 & Existen suficientes casos de prueba; normales, típicos, ejemplares, correosos, etc. & E \\ 
\hline 
TA & P6 & 10 & 9 & La tarea está bien estructurada y se entiende. & D \\ 
\hline 
TA & P7 & 10 & 9 & Sólo requiere habilidad cognoscitiva (no pericia física). & D \\ 
\hline 
TA & P8 & 9 & 8 & No se precisan resultados óptimos sino sólo Satisfactorios, sin comprometer el proyecto. & D \\ 
\hline 
TA & P9 & 9 & 8 & La tarea no requiere sentido común. & D \\ 
\hline 
TA & P10 & 7 & 9 & Los directivos están verdaderamente comprometidos con el proyecto. & D \\ 
\hline 
\end{tabular} \\

\begin{center}
\[
CV_{pl} = \prod_{i=1,2,5}(Vp_{i}//Vu_{i})[\prod_{j=1}^{10}Pp_{j}*Vp_{j}]^{1/10}
\]
$CV_{pl} = 75,965$
\end{center}

\newpage
\subsubsection{Justificación}
Podemos saber si está justificada la realización de un sistema por diversos
motivos, unos cuantos de estos podría ser: el sistema estaría justificado si los conocimientos o experiencia del experto esté en peligro de pérdida, podría no estar justificado si no se recupera el coste de su realización. \\

Obtención de valores para el cálculo de la justificación:


\begin{tabular}{|c|c|c|c|p{7.3 cm}|c|}
\hline 
CAT & IDEN. CAT & PESO & VALOR & DENOMINACIÓN DE LA CARACTERÍSITCA & TIPO \\ 
\hline 
EX & J1 & 10 & 10 & El experto NO está disponible. & E \\ 
\hline 
EX & J2 & 10 & 5 & Hay escasez de experiencia humana. & D \\ 
\hline 
TA & J3 & 8 & 9 & Existe necesidad de experiencia simultánea en muchos lugares. & D \\ 
\hline 
TA & J4 & 10 & 7 & Necesidad de experiencia en entornos hostiles, penosos y/o poco gratificantes. & E \\ 
\hline 
TA & J5 & 8 & 7 & No existen soluciones alternativas admisibles & E \\ 
\hline 
DU & J6 & 7 & 4 & Se espera una alta tasa de recuperación de la inversión & D \\ 
\hline 
DU & J7 & 8 & 8 & Resuelve una tarea útil y necesaria. & E \\ 
\hline 
\end{tabular} \\

\begin{center}
\[
CV_{ju} = \prod_{i=1,4,5,7}(Vj_{i}//Vu_{i})[\prod_{j=1}^{7}Pj_{j}*Vj_{j}]^{1/7}
\]
$CV_{ju} = 59,135$
\end{center}

\newpage
\subsubsection{Adecuación}
Se analiza si el problema puede resolverse con técnicas de ingeniería del conocimiento, se analiza su naturaleza, complejidad y tipo de tarea.\\

Obtención de valores para el cálculo de la adecuación: \\
\begin{tabular}{|c|c|c|c|p{7.3 cm}|c|}
\hline 
CAT & IDEN. CAT & PESO & VALOR & DENOMINACIÓN DE LA CARACTERÍSITCA & TIPO \\ 
\hline 
EX & A1 & 5 & 6 & La experiencia del experto está poco organizada & D \\ 
\hline 
TA & A2 & 6 & 10 & Tiene valor práctico. & D \\ 
\hline 
TA & A3 & 7 & 10 & Es una tarea más táctica que estratégica. No, ya que cubre necesidades actuales. & D \\ 
\hline 
TA & A4 & 7 & 9 & La tarea da soluciones que sirvan a necesidades a largo plazo. & E \\ 
\hline 
TA & A5 & 5 & 8 & La tarea no es demasiado fácil, pero es de conocimiento intensivo, tanto propio del dominio, como de manipulación de la información. & D \\ 
\hline 
TA & A6 & 6 & 8 & Es de tamaño manejable, y/o es posible un enfoque gradual y/o, una descomposición en subtareas independientes.Se espera poder ir añadiendo diferentes trastornos si fuese necesario. & D \\ 
\hline 
EX & A7 & 7 & 10 & La transferencia de experiencia entre humanos es factible (experto a aprendiz). & E \\ 
\hline 
TA & A8 & 6 & 7 & Estaba identificada como un problema en el área y los efectos de la introducción de un SE pueden planificarse. & D \\ 
\hline 
TA & A9 & 9 & 7 & No requiere respuestas en tiempo real inmediato. No es obligatorio que el resultado sea administrado en tiempo real aunque sea deseable. & E \\ 
\hline 
TA & A10 & 9 & 10 & La tarea no requiere investigación básica. & E \\ 
\hline
TA & A11 & 5 & 9 & El experto usa básicamente razonamiento simbólico que implica factores subjetivos. & D \\ 
\hline 
TA & A12 & 5 & 8 & Es esencialmente de tipo heurístico. & D \\ 
\hline 
 
\end{tabular} \\

\begin{center}
\[
CV_{ad} = \prod_{i=4,7,9,10}(Va_{i}//Vu_{i})[\prod_{j=1}^{12}Pa_{j}*Va_{j}]^{1/12}
\]
$CV_{ad} = 52,012$
\end{center}

\subsubsection{Éxito}
Se determina si el sistema tendrá exito atendiendo a las cuestiones planteadas en la siguiente tabla.\\

Obtención de valores para el cálculo del éxito:

\begin{tabular}{|c|c|c|c|p{7.3 cm}|c|}
\hline 
CAT & IDEN. CAT & PESO & VALOR & DENOMINACIÓN DE LA CARACTERÍSITCA & TIPO \\ 
\hline 
EX & E1 & 8 & 10 & No se sienten amenazados por el proyecto, son capaces de sentirse intelectualmente unidos al proyecto. & D \\ 
\hline 
EX & E2 & 6 & 10 & Tienen un brillante historial en la realización de esta tarea. & D \\ 
\hline 
EX & E3 & 5 & 8 & Hay acuerdos en lo que constituye una buena solución a la tarea. & D \\ 
\hline 
EX & E4 & 5 & 8 & La única justificación para dar un paso en la solución es la calidad de la solución final. & D \\ 
\hline 
EX & E5 & 6 & 5 & No hay un plazo de finalización estricto,ni ningún otro proyecto depende de esta tarea. & D \\ 
\hline 
TA & E6 & 7 & 10 & No está influenciada por vaivenes políticos. & E \\ 
\hline 
TA & E7 & 8 & 5 & Existen ya SS.EE. que resuelvan esa o parecidas tareas. & D \\ 
\hline 
TA & E8 & 8 & 7 & Hay cambios mínimos en los procedimientos habituales. & D \\ 
\hline 
TA & E9 & 5 & 10 & Las soluciones son explicables o interactivas. & D \\ 
\hline 
\end{tabular} \\

\begin{tabular}{|c|c|c|c|p{7.3 cm}|c|}
\hline 
CAT & IDEN. CAT & PESO & VALOR & DENOMINACIÓN DE LA CARACTERÍSITCA & TIPO \\ 
\hline 
EX & E10 & 7 & 9 & La tarea es de I+D de carácter práctico, pero no ambas cosas simultáneamente. & E \\ 
\hline 
EX & E11 & 6 & 10 & Están mentalizados y tienen expectativas realistas tanto en el alcance como en las limitaciones. & D \\ 
\hline 
EX & E12 & 7 & 10 & No rechazan de plano esta tecnología. & E \\ 
\hline 
EX & E13 & 6 & 9 & El sistema interactúa inteligente y amistosamente con el usuario. & D \\ 
\hline 
EX & E14 & 9 & 8 & El sistema es capaz de explicar al usuario su razonamiento. & D \\ 
\hline 
TA & E15 & 8 & 10 & La inserción del sistema se efectúa sin traumas; es decir, apenas se interfiere en la rutina cotidiana de la empresa. & D \\ 
\hline 
TA & E16 & 6 & 9 & Están comprometidos durante toda la duración del proyecto, incluso después de su implantación. & D \\ 
\hline 
TA & 17 & 8 & 9 & Se efectúa una adecuada transferencia tecnológica. & E \\ 
\hline 
\end{tabular} \\
\begin{center}
\[
CV_{ex} = \prod_{i=6,10,12,17}(Ve_{i}//Vu_{i})[\prod_{j=1}^{17}Pe_{j}*Ve_{j}]^{1/17}
\]
$CV_{ex} = 56,325$
\end{center}
\newpage
\subsection{Evaluación global las dimensiones de la aplicación}
Para la evaluación final extraemos la media de todas las aplicaciones candidatas: 
\begin{center}
\[
CV = \sum_{i=1}^{4}VC_{i}/4
\]
$CV = 60,85925$
\\
Porcentaje de viabilidad, normalización del resultado:
\begin{equation}
 \textit{Viabilidad} = \dfrac{CV*100}{CVmax} 
 \end{equation}\\

 CVmax = 76,21125 
 \begin{equation}
 \textit{Viabilidad} = \dfrac{61,28775*100}{76,21125} = 79,855 %
\end{equation}

\end{center}

El proyecto es viable.
\end{document}